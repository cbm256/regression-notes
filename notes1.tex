%%%%%%%%%%%%%%%%%%%%%%%%%%%%%%%%%%%%%%%%%
% Cheatsheet
% LaTeX Template
% Version 1.0 (12/12/15)
%
% This template has been downloaded from:
% http://www.LaTeXTemplates.com
%
% Original author:
% Michael Müller (https://github.com/cmichi/latex-template-collection) with
% extensive modifications by Vel (vel@LaTeXTemplates.com)
%
% License:
% The MIT License (see included LICENSE file)
%
%%%%%%%%%%%%%%%%%%%%%%%%%%%%%%%%%%%%%%%%%

%----------------------------------------------------------------------------------------
%	PACKAGES AND OTHER DOCUMENT CONFIGURATIONS
%----------------------------------------------------------------------------------------

\documentclass[11pt]{scrartcl} % 11pt font size

\usepackage[utf8]{inputenc} % Required for inputting international characters
\usepackage[T1]{fontenc} % Output font encoding for international characters

\usepackage[margin=0pt, landscape]{geometry} % Page margins and orientation

\usepackage{graphicx} % Required for including images
\usepackage{amsmath}
\usepackage{amssymb}
\usepackage{parskip}
\usepackage{color} % Required for color customization
\definecolor{mygray}{gray}{.75} % Custom color

\usepackage{url} % Required for the \url command to easily display URLs

\usepackage[ % This block contains information used to annotate the PDF
colorlinks=false, 
pdftitle={Cheatsheet}, 
pdfauthor={Aaron Hillegass}, 
pdfsubject={Compilation of useful shortcuts}
]{hyperref}

\setlength{\unitlength}{1mm} % Set the length that numerical units are measured in
\setlength{\parindent}{0pt} % Stop paragraph indentation

\renewcommand{\dots}{\ \dotfill{}\ } % Fills in the right amount of dots

\newcommand{\command}[2]{#1~\dotfill{}~#2\\} % Custom command for adding a shorcut

\newcommand{\sectiontitle}[1]{\paragraph{#1} \ \\} % Custom command for subsection titles

%----------------------------------------------------------------------------------------

\begin{document}

\begin{picture}(297,210) % Create a container for the page content


%----------------------------------------------------------------------------------------
%	FIRST COLUMN SPECIFICATION
%----------------------------------------------------------------------------------------

\put(5,205){ % Divide the page
\begin{minipage}[t]{92mm} % Create a box to house text

%----------------------------------------------------------------------------------------
%	HEADING ONE
%----------------------------------------------------------------------------------------

\paragraph{Regression} Let's say a given system has $p$ inputs and one output. Looking at the the historical inputs $\{ x_1, \ldots, x_n \}$ and the corresponding outputs $\{ y_1, \ldots, y_n \}$, we would like to make a guess what $y_i$ will be for an a new $x_i$.

\paragraph{Simple Linear Regression} In simple linear regression, there is only one input and our guess of $y_i$ will be given by the following:
\begin{equation*}
y_i = \beta_0  + \beta_1 x_i + \epsilon_i 
\end{equation*}
Where $E(\epsilon_i) = 0$ and the variance of $\epsilon$ is $\sigma_2$.

The mean of the observed inputs is
\begin{equation*}
\bar{x} = \frac{\sum_{i=1}^{n} x_i}{n}
\end{equation*}

The mean of the observed outputs is
\begin{equation*}
\bar{y} = \frac{\sum_{i=1}^{n} y_i}{n}
\end{equation*}


\paragraph{Least Squared Estimate (SLR)} 
\begin{equation*}
S_{xy} = \sum_{i=1}^{n} (x_i - \bar{x})(y_i - \bar{y}) 
\end{equation*}
\begin{equation*}
S_{xx} = \sum_{i=1}^{n} (x_i - \bar{x})^2 
\end{equation*}
\begin{equation*}
\hat{\beta_1} = \frac{S_{xy}}{S_{xx}} 
\end{equation*}

Using $\beta_1$, you can estimate $\beta_0$:
\begin{equation*}
\hat{\beta_0} = \bar{y} - \hat{\beta_1}\bar{x}
\end{equation*}

The prediction for $x_i$ is
\begin{equation*}
\hat{y_i} = \bar{\beta_0} + \hat{\beta_1}x_i
\end{equation*}

The prediction error (or residual) is
\begin{equation*}
\hat{\epsilon_i} = y_i - \hat{y_i}
\end{equation*}



\end{minipage} % End the first column of text
} % End the first division of the page

%----------------------------------------------------------------------------------------
%	SECOND COLUMN SPECIFICATION 
%----------------------------------------------------------------------------------------

\put(100,205){ % Divide the page
\begin{minipage}[t]{92mm} % Create a box to house text

\paragraph{ANOVA}
\begin{equation*}
SST = S_{yy} = \sum_{i=1}^{n} (y_i - \bar{y})^2  = SSR + SSE
\end{equation*}
(SST has $n-1$ degrees of freedom) Note that
\begin{equation*}
SSR = \hat{\beta_1} S_{xy} = SST - SSE
\end{equation*}
(SSR has 1 degree of freedom) and
\begin{equation*}
SSE = \sum_{i=1}^{n} \hat{\epsilon_i}^2  = SST + SSR
\end{equation*}
(SSE has $n -2$ degrees of freedom).

\begin{equation*}
MSR = \frac{SSR}{df(SSR)}
\end{equation*}

\begin{equation*}
\hat{\sigma}^2 = MSE = \frac{SSE}{df(SSE)}
\end{equation*}

\begin{equation*}
F statistic = \frac{MSR}{MSE}
\end{equation*}
which follows Snedecor's F-distribution with $df_1 = df(SSR)$ and $df_2 = df(SSE)$. The p-value is the tail probability of the observed F-statistic. Anything smaller than 0.05 is pretty good.

\begin{equation*}
R^2 = \frac{SSR}{SST} = 1 - \frac{SSE}{SST} = \hat{\beta_1}^2 \frac{S_{xx}}{S_{yy}} = \hat{\beta_1}\frac{S_{xy}}{S_{yy}}
\end{equation*}

\paragraph{Quality of Parameters}
The standard error of our estimate of $\hat{\beta_1}$ is
\begin{equation*}
s.e.(\hat{\beta_1}) = \sqrt{\frac{\hat{\sigma}^2 }{S_{xx}}}
\end{equation*}
The T-statistic for $\hat{\beta_1}$:
\begin{equation*}
\frac{\hat{\beta_1}}{s.e.(\hat{\beta_1})}
\end{equation*}
which follows Student's distribution with $df = n - 2$. The p-value is the tail probability of the observed t-statistic. Once again, anything smaller than 0.05 is pretty good.

\end{minipage} % End the second column of text
} % End the second division of the page

%----------------------------------------------------------------------------------------
%	THIRD COLUMN SPECIFICATION 
%----------------------------------------------------------------------------------------

\put(195,205){ % Divide the page
\begin{minipage}[t]{94mm} % Create a box to house tex

\paragraph{Confidence Interval of Expectation} The prediction of the mean response at $x = x_0$ is given by
\begin{equation*}
E(Y) = \hat{y_0} = \hat{\beta_0} + \hat{\beta_1} x_0
\end{equation*}


The standard error of the prediction of $E(Y)$ at $x_0$ is given by
\begin{equation*}
s.e.(prediction) = \sqrt{\hat{\sigma}^2 \left( \frac{1}{n} + \frac{x_0 - \bar{x}^2}{S_{xx}}\right)}
\end{equation*}

Thus, $100(1-\alpha)$ confidence interval of $E(Y)$ at $x = x_0$ is 
\begin{equation*}
Point Predition \pm \left( t_{\alpha/2, df=n-2}\right) \left(s.e.(prediction) \right)
\end{equation*}

\paragraph{Confidence Interval of New Observation} The prediction is the same. But the standard error is bigger:
\begin{equation*}
s.e.(prediction) = \sqrt{\hat{\sigma}^2 \left( 1 + \frac{1}{n} + \frac{x_0 - \bar{x}^2}{S_{xx}}\right)}
\end{equation*}

The confidence interval is calculated the same as above using the t-distribution.

\paragraph{Adjusted $R^2$}
\begin{equation*}
R_{adj}^2 = 1 - \left(1 - R^2\right)\frac{n - 1}{n - k - 1}
\end{equation*}
where $k$ is the number of parameters.

\paragraph{Variance Inflation Factor}
For a variable $X_j$ that is suspected of being correlated with other variables, we remove it if its VIF is greater than 5.
\begin{equation*}
VIF(X_j) = \frac{1}{1 - R_{j}^2}
\end{equation*}
where $R_j^2$ is the $R^2$ of the regression run without $X_j$.

\paragraph{MLR in matrices} Let $X$ be the matrix where the inputs for each sample are a row and the first item in the row is 1. Let $Y$ be the column vector of outputs. Let $\beta$ be the column vector of coefficients. Let $\Sigma$ be a column vector of residuals.
\begin{equation*}
Y = X\beta + \Sigma
\end{equation*}


%----------------------------------------------------------------------------------------

\end{minipage} % End the third column of text
} % End the third division of the page
\end{picture} % End the container for the entire page

%----------------------------------------------------------------------------------------

\end{document}